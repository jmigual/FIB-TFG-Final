% !TEX root = ../main.tex

\secc{Conclusions}

In this project a model that it's able to obtain a better \gls{CI} than the previous 
state-of-the-art values has been created. The model used a deep learning model to do so
and it used the extracted scalar radiomic features from the patients' \gls{CT} scans.

Although the final model used only the extracted scalar values, there's still room for 
improvement when using the \glspl{CNN}. This model was built and tested but a proper 
hyper-parameter search may get better results, future work will be directed towards
finding a good set of hyper-parameters that can help the model to converge.

Furthermore, even having access to \gls{CC} clusters and being able to use \glspl{GPU}
to train the deep learning models, the training times where very long so the developing
process was slow.

During the project some of the time was spent in learning how \glspl{DNN} and \glspl{CNN} 
work and how to use them for \gls{ML}. Moreover, during the pre-process task getting 
comfortable with the dataset and how to work with images and survival data was more
important than it seemed.

This project incorporates two important changes from previous approaches to survival
problems. The first one is a way to convert the regression problem into a classification
problem by using a siamese network. The second one is the usage of a \gls{DNN} to train
the model on survival data.

Even though the \gls{CI} has been improved, future work should also be focused in 
using the results into a patient's survival prediction 

Looking at the initial objectives at \autoref{sec:objectives} they all look complete,
although more work is still needed when using deep learning-based methods to analyze 
the images, such as \glspl{CNN} where there's still room for improvement.

