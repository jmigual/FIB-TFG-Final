% !TEX root = ../main.tex

\secc{Conclusions}

In this project a model that it's able to obtain a better \gls{CI} than the previous 
state-of-the-art values has been created. The model used a deep learning model to do so
and it used the extracted scalar radiomic features from the patients' \gls{CT} scans.

However, the model for which the best results were obtained did not use \glspl{CNN} to 
predict the \gls{CI}. When creating a model using \glspl{CNN} the training times were
really high so the development process was slow. Having access to \gls{CC} clusters  
and being able to use \glspl{GPU} train the models helped in obtaining a initial solution,
which, otherwise, could not have been obtained using only \acrshort{CPU} training.

Moreover, by using \gls{CC} clusters, it was possible to validate the scalar only model 
by using \gls{LOOCV} since all the folds were trained simultaneously in the cluster.

During the project some of the time was spent in learning how \glspl{DNN} and \glspl{CNN} 
work and how to use them for \gls{ML}, acquiring this knowledge was later found useful
when creating the different deep learning models. Moreover, during the pre-process task 
getting comfortable with the dataset and how to work with images and survival data was more
important than it seemed.

This project incorporates two important innovations from previous approaches to survival
problems. The first one, is a way to convert the regression problem into a classification
problem by using a siamese network. The second one is the usage of a \gls{DNN} to train
the model on survival data.

Looking at the initial objectives at \autoref{sec:objectives} they all look complete,
although more work is still needed when using deep learning-based methods to analyze 
the images, such as \glspl{CNN} where there's still room for improvement.

\ssecc{Future steps}

Some of the work previously stated can be improved in some ways. 

\begin{itemize}
  \item When working with the deep residual siamese and using \glspl{CNN} there's still 
  room for improvement. This model was built and tested but a proper 
  hyper-parameter search may get better results, future work will be directed towards
  finding a good set of hyper-parameters that can help the model to converge.
  \item Improvements towards using \glspl{CNN} for the model can also be focused in 
  using transfer learning once 3D pre-trained \gls{CNN} models start to appear.
  \item Even though the \gls{CI} has been improved, future work should also be focused in 
  using the results to predict a patient's survival time or risk.
  \item The found model should also be validated by using more data or an external dataset.
\end{itemize}
