% !TEX root = ../main.tex

\secc*{Abstract}
\markboth{ABSTRACT}{ABSTRACT}

This project is centered around the survival analysis problem. Its objective
is to create a model that is able to predict the survival outcome of Princess 
Margaret Cancer Centre patients.

To do so, a deep learning model has been used in combination from radiomic features
extracted from Computed Tomography scans together with the clinical information
provided for each patient.

Radiomics is a recent is a new field, relying on pre-defined, 
hand-engineered features computed from medical images, to better characterize tumours 
and predict survival outcome.

\secc*{Resum}

Aquest projecte es centra al voltant del problema d'anàlisis de supervivència. L'objectiu
és el de crear un model que sigui capaç de predir la capacitat de supervivència dels 
pacients del Princess Margaret Cancer Centre.

Per fer-ho, s'ha creat un model que usa \emph{deep learning} en combinació amb les característiques
radiòmiques extretes de Tomografies Computaritzades en conjunt amb la informació clínica
obtinguda de cada pacient.

La radiomèdia és un nou camp, depenent de característiques manualment predefinides computades 
d'imatges mèdiques, especialitzat en caracteritzar tumors i predir les probabilitats de
supervivència.


\secc*{Resumen}

Este proyecto está centrado alrededor del problema de análisis de supervivencia. Su objectivo
es el de crear un modelo capaz de predecir la capacidad de supervivencia de los pacientes
del Princess Margaret Cancer Centre.

Para hacerlo, se ha creado un modelo que usa \emph{deep learning}, en combinación con las 
características radiómicas extraídas de Tomografías Computarizadas en conjunto con la 
información clínica proporcionada del paciente.

La radiomedia es un nuevo campo, que se basa en características manualmente predefinidas 
computadas de imágenes médicas, para caracterizar mejor los tumores y predecir las probabilidades 
de supervivencia.